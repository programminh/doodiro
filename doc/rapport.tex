\documentclass[10pt]{article}

\usepackage{times}
\usepackage[utf8]{inputenc}
\usepackage[french]{babel}
\usepackage{amsmath}
\usepackage{amsfonts}
\usepackage{amssymb}
\usepackage{graphicx}
\usepackage{parskip}
\usepackage{multicol}
\usepackage{url}


\begin{document}


\title{IFT3225 - Rapport TP3}
\author{Vincent Foley-Bourgon (FOLV08078309) \\
Truong Pham (PHAL29018809)}
\maketitle

\section{Introduction}

Pour ce TP3, on devait écrire une application web qui réplique le
fonctionnement du logiciel doodle.com.  Nous avons utilisé le
porte-manteau ``doodiro'' pour notre application.

L'application est écrite en PHP et JavaScript avec une base de données
MySQL.

\section{Doodiro}

Doodiro est un logiciel très simple, dont voici une liste des
fonctionnalités:

\begin{itemize}
\item Authentification des utilisateurs
\item Ajout par n'importe quel utilisateur d'un événement
\item Le choix des invités pour un événement se fait de façon
  interactive et intuitive en utilisant JavaScript.
\item L'interface pour permettre aux invités de choisir leurs
  disponibilités est facile d'utilisation et utilise également du
  JavaScript pour améliorer son utilisabilité.
\item L'administrateur de Doodiro peut ajouter et supprimer des
  usagers.  Il peut également supprimer des événements.
\end{itemize}

\subsection{Utilisateurs}

Afin de tester les différentes fonctionnalités de Doodiro, nous
fournissons 4 comptes:


\begin{tabular}{|c | c | c |}
  \hline
  Email & Mot de passe & Administrateur? \\
  \hline
  foleybov@iro.umontreal.ca & foleybov & $\checkmark$ \\
  phamlemi@iro.umontreal.ca & phamlemi & ~ \\
  vaudrypl@iro.umontreal.ca & vaudrypl & ~ \\
  lapalme@iro.umontreal.ca & lapalme & ~ \\
  \hline
\end{tabular}

\subsection{Validation}

La validation dans les différents formulaires est faites via
JavaScript afin d'offrir aux usagers une expérience d'utilisation plus
prompte. Si on devait supporter des navigateurs plus anciens, ou si on
désirait avoir une protection supplémentaire, il faudrait dupliquer la
logique de validation en PHP.


\section{Choix technologies}

Malgré que le choix de PHP et MySQL soient imposés, nous avions une
certaine liberté des choix technologiques. Voici un résumé de ces
choix.

\subsection{Bootstrap}

La compagnie Twitter a créé un ``framework'' CSS appelé Bootstrap qui
permet de rapidement, et sans maux de têtes, créer de belles
interfaces web.  Le CSS est écrit de façon portable, ce qui nous a
évité les problèmes habituels de tester sur Firefox, Chrome, Safari,
Internet Explorer, etc.  De plus, ce framework donne une interface
connue à notre application que beaucoup d'utilisateurs apprécieront.

\subsection{PHP + MySQL}

Une problématique très connu de l'utilisation des bases de données
dans les applications web sont les injections SQL, où un utilisateur
``injecte'' dans une requête SQL du code pour se donner des droits
administrateurs, effacer la base de données, etc.  Une solution facile
à utiliser pour contrer ces problèmes sont les requêtes préparées, qui
sont disponibles dans le module MySQLi de PHP.  Nous verrons dans la
section ``Problèmes et solution'' que cette approche se termina en
échec.

\subsection{MySQL}

Bien que MySQL offre beaucoup de flexibilité dans la création de ses
tables, nous avons appliqué des techniques solides de base de données
dans la création de notre schéma. Notre schéma est 3NF, qui est
considérée comme étant la forme normale la plus intéressante à avoir
dans les applications, car elle offre un bon équilibre entre
l'intégrité référentielle et la facilité d'implémentation.

Afin d'augmenter la sécurité de notre application, les mots de passe
sont stockés en SHA1 dans la base de données plutôt qu'en clair.  Afin
de ne pas trop compliquer la programmation de Doodiro, nous n'avons
pas ajouté de ``salt strings''.


\section{Problèmes et solutions}

Ce TP a été rempli de problèmes, d'embûches et de ``hacks'' dûs à
l'utilisation de vieilles technologies PHP (5.0.4) et MySQL (4.1.20)
sur les serveurs du DIRO.  La majorité du temps de développement n'a
pas été passé dans l'ajout de nouvelle fonctionnalités, mais dans la
ré-écriture de code existant dans un style plus ancien, moins
modulaire et moins sécure afin d'être compatible avec l'environnement
de déploiement.

\subsection{Problème d'encodage de la base de données}

Malgré que tous les niveaux de notre applications soient en UTF-8
(pages HTML, code PHP et JavaScript, base de données), il y avait une
composante qui était en latin1, le lien {\em mysqli} entre notre
application PHP et notre base de données MySQL. Étant donné qu'il y
avait une différence d'encodage, les caractères accentués dans notre
base de données apparaissaient comme le caractères de remplacement
dans nos pages web. Comble de malheur, la méthod {\em
  \$mysqli->set\_charset} est apparu seulement dans la version 5.0.5
de PHP, donc impossible de s'en servir. À force de chercher sur
Internet, on a fini par trouver la commande {\em \$mysqli->query(``SET
  NAMES utf8'')} qui permet de régler le problème.

\subsection{Bugs dans mysqli}

Nous avons initialement choisis d'utiliser le module mysqli dans PHP
pour effectuer nos requêtes à la base de donnés, car contrairement au
module mysql, il permet de créer des requêtes préparées où les
procédures d'échappement sont laissées au soin de la librairie.
Malheureusement, cette fonctionnalité dans PHP 5.0.4 a un bug (elle
retourne des entiers aléatoires pour les clés primaires de tables) et
donc nous avons dû abandonner la sécurité des requêtes préparées et
utiliser le remplacement de chaîne bête, nous exposant ainsi à toutes
sortes d'injections SQL possibles.

Un autre bug dans mysqli se trouve dans la méthode {\em
  real\_escape\_string}; elle nous retournait toujours une chaîne vide
plutôt qu'une chaîne où les caractères spéciaux étaient échappés. Nous
avons dû nous résoudre à utiliser la fonction moins sécure {\em
  addslashes()}.

\subsection{Absence de json\_encode et json\_decode}

Notre application fait un usage très fort de technologies JavaScript,
et la plupart des données générées par notre code JS est du JSON. Les
versions récentes de PHP possèdent des fonction pour encoder et
décoder le JSON. Malheureusement, une fois de plus la version
préhistorique de PHP qui est installée au DIRO nous a causé problème,
car elle ne possède pas cette fonctionnalité. Nous avons donc dû nous
résoudre à utiliser des fonctions d'encodage et de décodage de qualité
inférieure qui ont été trouvées sur des forums d'aide de PHP.

\subsection{No framework}

Dans toutes les applications PHP modernes, le développement se fait à
l'aide d'un framework, habituellement MVC, qui permet de bien faire la
séparation de la logique et du contenu.  De plus, ces frameworks
réduisent la complexité du développement en s'occupant des tâches
mineures que l'on devrait normalement faire.  Écrire une application
en PHP pur s'est avéré être une tâche frustrante, plus longue que
nécessaire et le résultat final est un produit qui est moins
modularisé et de moins bonne qualité.




\end{document}
